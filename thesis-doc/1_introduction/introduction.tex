\chapter{Introduction}

% the code below specifies where the figures are stored
\ifpdf
    \graphicspath{{1_introduction/figures/PNG/}{1_introduction/figures/PDF/}{1_introduction/figures/}}
\else
    \graphicspath{{1_introduction/figures/EPS/}{1_introduction/figures/}}
\fi


\section{Music Information Retrieval \& Audio Fingerprinting} % 

Music information retrieval deals with the extraction, processing and interpretation of audio data. Tasks range from simple endeavors like feature extraction, filtering and watermarking all the way to the difficult and not-yet-solved problems like music transposition, instrument detection, cover song detection and music recommendation.\cite{mir}
This paper focuses on expanding an existing audio fingerprinting method with a music recommendation system.
 
\section{Intellectual Property Controversy} 

The algorithm used in this projects' implementation is less-commonly referred to as the \textbf{Landmark-based algorithm} or Landmark for short. More commonly it is know as the algorithm behind the popular mobile app Shazam.\cite{shazam}

The Landmark algorithm is described in Avery Wang's excellently titled "An Industrial-Strength Audio Search Algorithm"\cite{avery_shaz}

Since publication the Shazam Entertainment Limited holds a patent on the algorithm. Hence any open-source implementation of Wang's described algorithm falls within a legal gray area.

% consider adding shazam patent stuff https://www.shazam.com/gb/patents

One such example is Roy van Rijn's Java implementation and subsequent cease and desist letter due to patent infringement. This occurs in 2010. \cite{roy_infringement}

Similarly, Milos Miljkovic describes the same legal quandaries five years later in his appearance at the PyData conference in New York City. \cite{milos}

Intellectual property issues aside the algorithm has seen tremendous attention from the open source community. Numerous implementations in different stages of completion are available on public repositories.

\section{Motivation}
Albeit an interesting trivia item, the commercial presence of Shazam is not a focus of this project. 

The author's motivation in building on top of the Landmark algorithm is driven by curiosity and a desire to expand upon a known fingerpriting technique. 

Consequently, this project involves an implementation of the Landmark algorithm, an assessment of the algorithm efficiency and finally an extension to the original concept, followed by an assessment.






